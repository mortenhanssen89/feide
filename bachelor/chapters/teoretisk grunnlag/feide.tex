\section{Teoretisk grunnlag for Feide}
Feide er kunskapsdepartments løsning for sikker identifisering i utdanningssektoren. \\ \\
Feide er basert på konseptet føderert identidetishåntering. Dette vil si oppretting og autentisering av brukere skjer hos vertsorganisasjonene . Vertsorganisasjonene har også ansvar for administering av rettigheter, passord og personopplysninger. Tjenestetilbyder tar seg ikke av brukeradministrasjon, men administrerer rettigheter til tjenestene. \\ \\
\begin{description}
	\item[\tab •] Single Sign-On (SSO) er en måte å minske risiko for menneskelig feil. Brukeren vil da ha tilgang til flere program, tjenester eller datamaskiner ved bruk av samme brukernavn og passord.
	\item[\tab •] Single LogOut (SLO) er for å rydde opp. Hvis brukeren vil logge inn i en annen tjeneste så får brukeren oversikt over hvilke andre tjenester som man allerede er logget inn på. Uansett hvilken tjeneste og hvor man befinner seg så kan man logge ut av de andre tjenestene.
\end{description}
Føderert identifiseringshåndtering skaper et tillitsforhold mellom vertorganisasjonen og tjenesten. Tjenesten bruker opplysningnene som  vertsorganisisasjonen sitter med angående tilgangskontrollen og personidentifiseringen. \\
For at tjenestene fortsatt skal ha kontroll over de viktigste avgjørelsene så bruker Feide denne tilnærmingen:
\begin{description}
	\item[\tab •] Vertsorganisasjoner registrerer og autentiserer sine brukere.
	\item[\tab •] Tjenesteleverandører bestemmer egne tilgangsregler.
\end{description}
