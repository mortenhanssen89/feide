\documentclass[12pt,a4paper]{article}
\usepackage[left=20mm,top=20mm,bottom=30mm]{geometry}

\begin{document}
\title{Forprosjekt \\ for \\ SkyHiGH Feide}
\author{Fredrik Magnussen, Håkon Tvedt og Morten Hanssen Singstad}
\maketitle

\newpage
\tableofcontents

\newpage
\subsection{Planlegging, oppfølging og rapportering}
\subsubsection{Hovedinndelinger av prosjektet}
Første del av prosjektet, vil gå til å lese gjennom de foregående SkyHiGH-prosjektene, for å kunne forstå hva som er 
gjort tidligere. Med dette får vi vite enda mer hvordan vi skal sette opp igjen skyløsningen for VM-er med Openstack.
Openstack har også kommet med en ny oppdatering, derfor må vi også finne ut om vi må gjøre endringer, med tanke på installeringen 
av SkyHiGH Adm delene. Siden vi foreløpig må vente på strøm og Internett til serverne som vi skal bruke, så blir vi nødt til å 
sette opp og teste Openstack med SkyHiGH Adm i en VM på vår egen maskin. \newline \newline
Med dette så kan vi få begynt på del to. Siden vi er driftere så vil følge en uskreven regel som er; Kan en oppgave automatiseres, så skal den automatisere.
Med dette så skal vi lage script. Script for installering av OS på serverne. Script for å ta backup som et image av OS-et, med de siste konfigurasjonene og installeringer.
Dette er også for at det skal spare oss for problemer. Hvis vi skulle være så uheldige med f.eks systemet vårt, så vil vi alltid ha en fungerende installasjon, uten å miste
altfor mye arbeid. 

\subsection{Organisering og kvalitetssikring}

\subsection{Plan for Gjennomføring}

\end{document}